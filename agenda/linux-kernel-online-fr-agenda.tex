\documentclass[a4paper,12pt,obeyspaces,spaces,hyphens]{article}

\def \trainingtype{online}
\def \agendalanguage{french}

\input{agenda/linux-kernel.inc}

\usepackage{agenda}

\begin{document}

\feshowtitle

\feshowinfo

\showboarditem{beagleboneblack}
\showboarditem{beagleplay}

\feagendaonecolumn
{Travaux pratiques}
{
  Les travaux pratiques de cette formation font appel aux périphériques
  matériels suivants, pour illustrer le développement de pilotes de
  périphériques pour Linux :

  \begin{itemize}
  \item Une manette Nunchuk pour console Wii, qui est connectée à la
    BeagleBone Black via le bus I2C. Son pilote utilisera le
    sous-système {\em input} du noyau Linux.
  \item Un port série (UART) supplémentaire, dont les registres sont
    mappés en mémoire, et pour lequel on utilisera le sous-système {\em
    misc} de Linux.
  \end{itemize}

  Bien que nos explications cibleront spécifiquement les sous-systèmes
  de Linux utilisés pour réaliser ces pilotes, celles-ci seront toujours
  suffisamment génériques pour faire comprendre la philosophie
  d'ensemble de la conception du noyau Linux. Un tel apprentissage
  sera donc applicable bien au delà des périphériques I2C, d'entrée ou
  mappés en mémoire.
}

\onlineagenda

\end{document}

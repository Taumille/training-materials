\documentclass[a4paper,12pt,obeyspaces,spaces,hyphens]{article}

\def \trainingtype{online}
\def \agendalanguage{english}

\input{agenda/linux-kernel.inc}

\usepackage{agenda}

\begin{document}

\feshowtitle

\feshowinfo

\showboarditem{beagleboneblack}
\showboarditem{beagleplay}

\feagendaonecolumn
{Labs}
{
  The practical labs of this training session use the following
  hardware peripherals to illustrate the development of Linux device
  drivers:

  \begin{itemize}
  \item A Wii Nunchuk, which is connected over the I2C bus to the
    BeagleBone Black board. Its driver will use the Linux {\em input}
    subsystem.
  \item An additional UART, which is memory-mapped, and will use the
    Linux {\em misc} subsystem.
  \end{itemize}

  While our explanations will be focused on specifically the Linux
  subsystems needed to implement these drivers, they will always be
  generic enough to convey the general design philosophy of the Linux
  kernel. The information learnt will therefore apply beyond just
  I2C, input or memory-mapped devices.
}

\onlineagenda

\end{document}

